\section{Algorithm to Construct a Lower Bound $m(d, k)$}

We now provide an algorithm that takes as an input a set of generators and two sets of polynomial coefficients. These inputs are used to test the ability of a generated lower bound to provide a set covering for our Cayley Graph. The algorithm iterates through permutations of the generating set and possible coverings, giving as an output the maximal generating set as well as the optimal polynomial bound. \n
The algorithm takes advantage of the fact that there exists a one-to-one correspondance between representations of the residual classes of $\Z_m$ and the set of integers $[0, m-1]$.\n

\subsection{Polynomial Construction}
\noindent
Let $d$ be the diameter for $\Cay(m, \mathscr{A})$.\n 

\subsubsection{Permutations of Generators and Coefficients}
The algorithm producing generators and coefficients with which the lower bound is constructed is the heart of functionality for the algorithm. \n
The permutations tested may be stored as data structures or file tables.

\noindent
Given $d$ define \emph{\bf{$d_{1}$ fixed}} to be $\frac{d}{\lambda}$. This will be the $d$ referred to in the algorithm.\n

The purpose of the following data structures is to provide containers for the following:\n
Define $\mathscr{A}$ to be a \emph{set of generators}, $\mathscr{A} = \{ (a_{i}) \vert a_{i+1} = \alpha_{i}a_{i},  \forall i \in [0, k-1] \} $, with $a_i$ a sequence of nonnegative coefficients, and $\vert \mathscr{A} \vert = k$.\n
Define a \emph{set of non-negative coefficients} $c_1, c_2, .. , c_k$ such that $a_{i+1} = \alpha_{i}a_{i} \lambda$, where ($\alpha_{i}$) is any sequence of coefficients.\n
Lastly, define a \emph{second set of non-negative coefficients} $x_1, x_2, .., x_{k}$ such that $x_{i+1} < d - \sum_{0}^{i}x_i$.
\noindent
Assume that the set of generator combinations is defined as an upper-triangular two-dimensional array of unordered coefficients represented by tuples, the sequences of coefficients are defined as ordered lists of tuples.

\subsubsection*{Tuple Data Structure}

\begin{lstlisting}

template<TP, N>
class Tuple{

    vector<TP> data; // class contains an N-vector of type TP

    ...
};

\end{lstlisting}

\subsubsection*{Construction of the Coefficient Permutations}

\begin{lstlisting}

typedef Tuple<int, k> T;


// Construction of X-Coefficients
for(x1 = d; x1 >= 0; --x1)
{
  for(x2 = d - x1; x2 >= 0; --x2)
  {
     for(x3 = d - x1 - x2; x3 >= 0; --x3) 
     {
        ...
            for(xk = d - x1 - x2 - x3 - .. - xk; xk >= 0; --xk)
            {
                 if(x1 + x2 + x3 + .. +xk <= d - k)
                 {
                      out << T(xk, xk-1, .. , x1); // In this case we are storing in a data file.
                      ++size;
                 }
            }
        ...
     }	
  }
}

// Construction of M-Coefficients
for(int c1 = 1; c1 < (d^k / (k! * a1 * a2 * .. * ak-1)); ++c1)
{
  for(int c2 = 1; c2 < a1; ++c2)
  {
     for(int c3 = 1; c3 < a2; ++c3)
     {
         ...
             for(ck = 1; ck < ak-1; ++ck)
             {

                 out << T(ck, ck-1, .. , c1);
                 ++size;
             }
         ...
     }
  }
}

// Construction of Generators
for(a1 = 2; a1 < (d^k / k!); a1++) // We are using the trivial upper bound to constrain the size of the generators.
{
  for(a2 = 2; a2 < (d^k / k!); a2++)
  {
        ...
             if(a1 * a2 * .. * ak-1 < d^k / k!)
             {
                 out<< T(a1 * a2 *.. * ak-1, a2 * a3 * .. ak-1, ... , ak-1, 1);
                 ++size;
             }
        ...
  }
}


\end{lstlisting}

\subsubsection{Polynomial Construction}

Our lower bound on $m(d, k)$ will be defined as $m(d, k) = a_{i}c_{i} \lambda$. To determine the validity of the lower bound, we compute every point in $d \mathscr{A}$ as a polynomial in terms of $\lambda$.\n

\begin{lstlisting}

Polynomial P(Tuple A, Tuple Y);

\end{lstlisting}

\noindent
($\lambda$ is a large number determined by $d$. The parameter $\lambda$ will not appear in the code, but it enables us to compute a lower bound as a function of $d$.)\n

\subsection*{Polynomial Data Structure} The basic polynomial data structure may be defined as follows:

\begin{lstlisting}

class Polynomial{

    Tuple A;  // container for Generators
    Tuple Y;  // container for M-Coefficients or X-Coefficients

    ...
};

\end{lstlisting}

In the case where Polynomial \lstinline{P.Y} is a container for the M-Coefficients, the class instance is a representation of the polynomial $x$. In the case where \lstinline{P.Y} is a container for the X-Coefficients, the class instance is a representation of $X'$.
\lstinline{P.Y[0]} will refer to the largest polynomial coefficient. \lstinline{P.A[0]} will refer to the largest corresponding generator.

\subsubsection{Constructing a Representative Class of Polynomials}

For all constructed polynomials $x = x_1a_{1} + x_2a_{2} + .. + x_{k}a_{k}$, we define a representative $x' \in [1, m-1]$ to which we will map all congruent polynomials, forming our residue class $\bar{x}$ of regular polynomials.\n

\begin{proposition} 
$ \{ (x_1, x_2, ... , x_k) \vert x_{1} \leq c_1, x_{2} \leq c_2, .. , x_{k} \leq c_k$, and $\sum_{i} x_i \leq d_{1} \}$ define polynomials $x$ that are considered to be \textbf{minimal}.\n
\end{proposition}

\begin{proposition}
The constructed polynomial $x$ is defined to be \textbf{regular} if $x \in [0, m-1]$.
\end{proposition}
We will use regular polynomials to determine whether or not $d \mathscr{A} = \Z_m$ by only considering a single covering of $\Z_m$.\n

\noindent
Note that a regular polynomial need not be minimal.\n
\noindent
We check for regularity by comparing the coefficients $(x_1, x_2, .. , x_{k})$ and $(c_1, c_2, .. , c_k)$ from their respective polynomials.\n

\begin{proposition}
$\forall x \in d \mathscr{A}$ if $x \notin [0, m-1]$, we can identify $x$ with point $x' = (x'_1, x'_2, .. , x'_k) \in [0, m-1]$ congruent to $x$ $(mod$ $m)$.  Then if every point $n \in \Z_m$ is either equal to some $x$ or $x'$, then $d \mathscr{A} \cong \Z_m$.\n
\end{proposition}

\noindent
Consider the case, $x > m(d, k)$, where $x$ is not regular. We perform a recursive polynomial subtraction of the coefficients where $c_1, c_2, c_3$ is subtracted term-by-term from $x_1, x_2, x_3, ... , x_k$. The resulting low-order coefficients are then forced to be positive by adding the generator associated with the next higher-order term.\n

\begin{lstlisting}

// Overloaded polynomial subtraction operator.
Polynomial Polynomial::operator-(Polynomial m)
{
    loop:
    while( Y > rhs.Y )
    {
          Y = T(Y[0] - rhs.Y[0], Y[1] - rhs.Y[1], Y[2] - rhs.Y[2], ... , Y[k] - rhs.Y[k]);
    }
    while( Y[1] < 0 )
    {
          Y = T(Y[0], Y[1] + (A[0] / A[1]), Y[2], ... , Y[k]);
    }
    while( Y[2] < 0 )
    {
          Y = T(Y[0], Y[1], Y[2] + (A[1] / A[2]), ... , Y[k]);
    }
    while( Y[3] < 0 )
    {
          Y = T(Y[0], Y[1], Y[2], Y[3] + (A[2] / A[3]), ... , Y[k]);
    }

    ...

    while( Y[k] < 0 )
    {
         Y = T(Y[0], Y[1], Y[2], Y[3] + (A[2] / A[3]), ... , Y[k] + (A[k-1] / A[k]));

    }
    if( Y > rhs.Y ){ goto loop; } 

    return *this;
}
\end{lstlisting}

\subsubsection{Construction of the Polynomial Bound}

To construct a lower bound, we systematically check combinations of generators $\mathscr{A}$ and coefficients, and record the largest $m$ (and corresponding generators) such that a covering by $d \mathscr{A}$ is achieved.\n
We will require that our polynomial representative of m to be regular, in order to ensure $x' \cong x$.

\begin{proposition}
Define $x'$ to be \textbf{regular} if $\forall x_i, x_i < c_{i+1}$.  
\end{proposition}

\begin{lstlisting}

\\ Define a class function to return m, the summation of our polynomial bound.
TP Polynomial::value()
{
    return (a1*y1 + a2*y2 +   + ak*yk);
}

\\ Define a class function to check that x' is well-formed
bool Polynomial::wellFormed()const
{
       return (Y[1] < ( A[0] / A[1]) && Y[2] < (A[1] / A[2]) && .. && Y[k] < (A[k-1] / A[k]));
}

\end{lstlisting}

\begin{proposition}
$\forall n \in \Z_m$, $\exists  x$ such that $\bar{x} = \bar{x'}= n$, then $d \mathscr{A} \supseteq \Z_m$.\n
\end{proposition}

\noindent
We will use a boolean array in order to check the inclusion of $\Z_m$ in $d \mathscr{A}$. Note that in order for the polynomial comparison to function properly, the tuple comparisons should be done in lexigraphical order as for regular tuples, this corresponds to the actual polynomial order.

\begin{lstlisting}

Polynomial mbest; // Container for our best lower bound m.

vector<bool> cover;
T A; // Container for a generating set.
T Q; // Container for a set of M-Coefficients
T x; // Container for a set of X-Coefficients.
// NOTE: The A, Q, and X tuples are retrieved from a file or data structure.

for(i = 0; i < sizeof(generators) // Number of generating sets. ; ++i)
{
    T A(ak, ak-1, .., a1);

    for(j = 0; j < sizeof(mcoeffs) // Number of m-coefficients. ; ++j)
    {
         T Q(ck, ck-1, .., c1); 
 
         Polynomial M(A, Q); // Create initial polynomial representation of the lower bound.
   
         cover.clear(); 
         if((M.value() > mbest.value())) // Ignore M that are too small.
         {
            for(k = 0; k < sizeof(xcoeffs) // Number of x-coefficients. ; ++k)
            {
                T x(xk, xk-1, .., x1);

                Polynomial X(A, x); // Create our initial representative.
                Polynomial X_prime(X-M); // Enforce regularity.
                if(X_prime.wellFormed())
                { 
                    cover.push_back(1);
                }
            }
            if(accumulate(cover.begin(),cover.end(),0) == M.value())
            {
                mbest = M; // Store improved lower bound.
            }
         }
    }
}

\end{lstlisting}

\subsubsection*{Parallelization}

The looping nature of the program makes it a natural candidate for optimization using parallel processing techniques. In this case, we have used a cluster of computer nodes. The largest dataset that is iterated over is the M-Coefficients, and can be a "bottleneck" in the program. Parallelization solves this problem by allowing a number of nodes that compute extremely large datasets to compute for an extended time or even without completion. These larger computations will not interfere with nodes that are computing bounds based on subsequent generating sets. The effect of the multiple processes is that we are able to check a larger set of generators, in fact checking up to the current trivial lower bound of $\frac{d^k}{k!}$.


\subsection{Complexity of the Algorithm}

\subsubsection{Generators and Coefficients}

The complexity of the algorithm depends both on the number of generators as well as the chosen diameter, however since our program deals with smaller values of $k$, and our chosen diameter may be arbitrarily large, we will formulate our complexity analysis in terms of $d$. As we are checking all permutations of generators up to the upper bound $\frac{d^k}{k!}$, we are able to treat the coefficient $\frac{1}{(k!)^{k}}$ as a constant term and express the compexity as an average case in big-$\mathcal{O}$ notation. The number of operations per loop in the construction of the generators is on the order of $\mathcal{O}(d^{2k})$. The construction of the coefficients to generate our representative polynomial $m$ is dependent on the individual generating set for which it is being constructed. In the worst case, the generating set will have large generators, allowing for the outermost loop to finish quickly, however, this will make it possible for the inner loops to approach or surpass the complexity of the larger, outermost loop. The complexity in this case is $\mathcal{O}(\max_{a \in \mathscr{A}}(a \cdot d)^k)$, where for most $a \in \mathscr{A}$, $a < d$. In the best case, the generating set will have small coefficients, in which case the outermost loop will have complexity of $\Omega(\max_{a \in \mathscr{A}} (\frac{d}{a^k})^k)$, and the inner loops will finish quickly. The final complexity of the loop structure in this case can be described as $\Theta(d^k)$. Finally, the coefficients for our representative polynomial are generated in $\mathcal{O}(d^k)$.

\subsubsection{Polynomial Bound}

The generation of the polynomial bound is again a nested loop structure which checks each possible representative for each possible combination for each generating set that can be created where each generator is the largest possible multiple it can take for its position up to the trivial upper bound. We can describe the complexity of this main program as $\mathcal{O}(d^k) \cdot \Theta(d^k) \cdot \mathcal{O}(d^k) = \mathcal{O}(d^{3k})$. As we found, the size of the coefficient set for the polynomial bound depends largely on the specific generating set being examined, significantly affecting the complexity of the inner loops (or coefficient loops).
