\section{New Lower Bound of $m(2, k)$.}

Let $d$ be a positive integer. We define $n(d, k)$ to be the largest positive integer $m$ such that there exists a subset $A$ of positive integers with the property that every integer in the interval $[1, m]$ is the sum of at most $d$ not necessarily distinct elements of $A$. Comparing $m(d, k)$, we may say that $n(d, k)$ is the interval version of $m(d, k)$. In other words, 
\[
n(d, k) = \max_{\scriptstyle A\subseteq\mathbb Z^{+}\atop\scriptstyle |A|=k}\{m\  |\ d\cdot A\supseteq[1, m]\}.
\]
It is clear that for all $d \geq 1$ and $k \geq 1$ $m(d, k) \geq n(d, k)$. The computation of  $n(d, k)$ is often referred as the \emph{the postage stamp problem}. \emph{The postage stamp problem} is an old problem in number theory and has been studied extensively. See  for more information. 

The  best known lower bound for $n(2,k)$ was proved by Mrose in 1979:
\[
n(2, k) \geq \frac{2}{7}k^2 + \frac{12}{7}k + O(1)\qquad  \text{as}\quad  k \to \infty.
\]
It is clear that $m(d,k)\ge n(d,k)$ for all $d$ and $k$. Furthermore, $m(d,k)$ should be {\em much\/} larger than $n(d,k)$ as at least one of $d$ and $k$ is getting large.  But the only known lower bound for $m(2,k)$ is
\[
m(2,k)\ge \frac27k^{2}+O(k),
\]
which is obtained by using the obvious inequality $m(d,k)\ge n(d,k)$.  In this paper, we improve the lower bound for $m(2,k)$ in the following theorem.
 
 
\begin{theorem}
$\displaystyle m(2,k) \geq \frac{37}{121}k^2 + O(k)$  as $ k \to \infty$.
\end{theorem}

\begin{proof}
Let $k \geq$ 14 be an integer. Let $\displaystyle k_1 = \left \lfloor \frac{k - 9}{11} \right \rfloor$. Let $m = 37k_1^2$.  Define 
\begin{align*}
I_\mu &= [\mu k_1^2, \mu k_1^2 + k_1],
&&\mu = 0, 4, 15, 26; \\
S_\nu &= \{\nu k_1^2 + ik_1 \ |\  i = 0 , 1, ... , k_1 - 1\},
&&\nu = 0, 1, 2, 3;\\
T_{\eta} &= \{\eta k_1^2+ i(k_1 + 1) \ |\   i = 0 , 1, ... , k_1 -1\},
&&\eta = 10, 20, 30.
\end{align*}
 Let $S= S_{0} \cup S_{1} \cup S_{2} \cup S_{3}$, and
define
\[
A=I_0\cup I_4\cup I_{15}\cup I_{26}\cup S\cup T_{10}\cup T_{20}\cup T_{30}.
\]
Noting  that $I_0\cap S_0=\{0,k_1\}$, and 
\[
|I_\mu| = k_1 + 1,\quad 
|S_\nu| = k_1,\quad\text{and}\quad
|T_\eta| = k_1,
\]
we see that 
\[
|A| \leq 11k_1 + 2 \leq k.
\]  
We now prove that $A+A=\Z_m$.

We begin by claiming $I_{\mu}+T_{\eta}\supseteq[(\mu + \eta )k_1^2 ,  (\mu + \eta  + 1)k_1^2)$. Let $n \in[(\mu + \eta )k_1^2 ,  (\mu + \eta  + 1)k_1^2)$.
Then we can write $n$ as 
\[
n = (\mu + \eta ) k_1^2 + qk_1 + r,
\]
where $0 \leq q < k_1$ and $0 \leq r < k_1$.

If $r \geq q$, then $0\le r-q<k_{1}$ and
\[
n = \mu k_1^2 +  (r - q)+ \eta k_1^2 + q(k_1+1).
\]
 Since 
 \[
 \mu k_1^2 + (r - q) \in I_\mu\qquad\text{and}\qquad \eta k_1^2 +q(k_1+1) \in T_\eta,
 \]
we see that $n \in I_{\mu}+T_{\eta}$. 

If $r < q$, then we must have $q \geq 1$ and $0\le k_{1}+r-q+1\le k_{1}$. Then
\[
\mu k_1^2 + (k_1 + r - q + 1) \in I_\mu \qquad \text{and}\qquad
\eta k_1^2 + (q - 1)(k_1 + 1) \in T_\eta.
\]
 Hence
\[
n = \mu k_1^2 + (k_1 + r - q + 1)+\eta k_1^2 + (q - 1)(k_1 + 1) \in I_{\mu}+T_{\eta}.
\]
 

Next we claim that  $I_{\mu}+S_{\nu}\supseteq[(\mu + \nu )k_1^2 ,  (\mu + \nu  + 1)k_1^2)$. Let $n \in[(\mu + \nu )k_1^2 ,  (\mu + \nu  + 1)k_1^2)$.
Then we can write $n$ as 
\[
n = (\mu + \nu ) k_1^2 + qk_1 + r = \mu k_1^2 + r+\nu k_1^2 + qk_1,
\]
where  $0 \leq q < k_1$ and $0 \leq r < k_1$, such that 
\[
\mu k_1^2 + r \in I_\mu \qquad\text{and}\qquad\nu k_1^2 + qk_1 \in S_\nu,
\]
 so $n \in I_{\mu}+S_{\nu}$. 

Our final claim is $S+T_\eta \supseteq[(\eta  +1)k_1^2 ,  (\eta  + 4)k_1^2)$.  Let $n \in[(\eta  +1)k_1^2 ,  (\eta  + 4)k_1^2)$.
Then we can write $n$ as 
\[
n = (\eta  + \nu ) k_1^2 + qk_1 + r, 
\]
where $1 \leq \nu  \leq 3$, $0 \leq q < k_1$, and $0 \leq r < k_1$. 

If $q \geq r$, then
\[
n = \nu k_1^2 + (q - r) k_1+\eta k_1^2 + r(k_1 + 1), 
\]
where $\nu k_1^2 + (q - r)k_1 \in S_\nu$ and $\eta k_1^2 + d(k_1 + 1) \in T_\eta$, so $n \in S_\nu +T_\eta$. 

If $q <r$, then
\[
n = (\nu  - 1)k_1^2 + (k_1 + q - r) k_1+\eta k_1^2 + r(k_1 + 1), 
\]
where 
\[
(\nu  - 1)k_1^2 + (k_1 + q - r)k_1 \in S_{\nu  - 1}\qquad\text{and}\qquad
\eta k_1^2 + r(k_1 + 1) \in T_\eta.
\]
Hence $n \in S_{\nu-1}+T_\eta\subseteq S+T_{\eta}$. 


It is clear that, in $\Z_{37k_{1}^{2}}$,
\begin{align*}
[45k_{1}^{2},46k_{1}^{2})&=[8k_1^2, 9k_1^2),&
[46k_{1}^{2},47k_{1}^{2})&=[9k_1^2, 10k_1^2),\\
[56k_{1}^{2},57k_{1}^{2})&=[19k_1^2, 20k_1^2).
\end{align*}
Therefore, we have proved that the entire interval $[0, m)=\Z_{m}$ can be  {\em covered\/} as follows: 
\begin{align*}
I_0 + S &\supseteq [0, 4k_1^2),&
I_4 + S &\supseteq [4k_1^2, 8k_1^2),\\
I_{15} + T_{30}  &\supseteq[45k_{1}^{2},46k_{1}^{2})= [8k_1^2, 9k_1^2),&
I_{26} + T_{20}  &\supseteq[46k_{1}^{2},47k_{1}^{2})= [9k_1^2, 10k_1^2),\\
I_{0}+T_{10}  &\supseteq [10k_1^2, 11k_1^2),&
S+T_{10}  &\supseteq [11k_1^2, 14k_1^2),\\
I_{4} + T_{10}  &\supseteq [14k_1^2, 15k_1^2),&
I_{15} + S  &\supseteq [15k_1^2, 19k_1^2),\\
I_{26} + T_{30}  &\supseteq[56k_{1}^{2},57k_{1}^{2})= [19k_1^2, 20k_1^2),&
I_{0} + T_{20}  &\supseteq [20k_1^2, 21k_1^2),\\
S + T_{20}  &\supseteq [21k_1^2, 24k_1^2),&
I_{4} + T_{20}  &\supseteq [24k_1^2, 25k_1^2),\\
I_{15} + T_{10}  &\supseteq [25k_1^2, 26k_1^2),&
I_{26} + S  &\supseteq [26k_1^2, 30k_1^2),\\
I_{0} + T_{30}  &\supseteq [30k_1^2, 31k_1^2),&
S + T_{30}  &\supseteq [31k_1^2, 34k_1^2),\\
I_{4} + T_{30}  &\supseteq [34k_1^2, 35k_1^2),&
I_{15} + T_{20}  &\supseteq [35k_1^2, 36k_1^2), \\
I_{26} + T_{10}  &\supseteq [36k_1^2, 37k_1^2).
\end{align*}
It now follows that 
\[
A + A \supseteq [0, 37k_1^2).
\]
Hence
\begin{align*}
m(2,k) &\geq 37k_1^2\\
&= 37 \cdot \left \lfloor \frac{k - 2}{11}\right \rfloor^2\\
&> 37 \left(\frac{k - 2}{11} -1\right) ^2\\
&= \frac{37}{121}k^2-\frac{962}{121}k+\frac{6253}{121}\\
&= \frac{37}{121}k^2 + O(k)\qquad \text{ as }\quad k \to \infty. \qedhere
\end{align*}
\end{proof}

