\section{ New Lower Bound of $m(2, k)$.}

Let $d$ be a positive integer. We define $n(d, k)$ to be the largest positive integer $m$ such that there exists a subset $A$ of positive integers with the property that every integer in the interval $[1, m]$ is the sum of at most $d$ not necessarily distinct elements of $A$. Comparing $m(d, k)$, we may say that $n(d, k)$ is the interval version of $m(d, k)$. In other words, 
\[
n(d, k) = \max_{A\subseteq\Z^{+}\atop\scriptstyle |A|=k}\{m\  |\ d\cdot A\supseteq[1, m]\}.
\]
It is clear that for all $d \geq 1$ and $k \geq 1$ $m(d, k) \geq n(d, k)$. The computation of  $n(d, k)$ is often referred as the \emph{the postage stamp problem}. \emph{The postage stamp problem} is an old problem in number theory and has been studied extensively. See  for more information. 

The following theorem was proved by Mrose in 1979, and gives the current bound for $n(2, k)$.
\[
n(2, k) \geq \frac{2}{7}k^2 + \frac{12}{7}k + O(1)\  \text{as} \  k \to \infty.
\]

 
In this paper we prove the following lower bound for $m(d, k)$.
\begin{theorem}
$\displaystyle m(2,k) \geq \frac{37}{121}k^2 + O(k)$  as $ k \to \infty$.
\end{theorem}

\begin{proof}
Let $k \geq$ 14 be an integer. Let $k_1 = \left \lfloor \frac{k - 9}{11} \right \rfloor$. Let $m = 37k_1^2$.  Define 
\begin{align*}
I_\mu &= [\mu k_1^2, \mu k_1^2 + k_1], \qquad
\mu = 0, 4, 15, 26; \\
S_\nu &= \{\nu k_1^2 + ik_1 \ |\  i = 0 , 1, ... , k_1 - 1\},\qquad
\nu = 0, 1, 2, 3;\\
T_{\eta} &= \{\eta k_1^2+ i(k_1 + 1) \ |\   i = 0 , 1, ... , k_1 -1\},\qquad
\eta = 10, 20, 30.
\end{align*}
 Let $S= S_{0} \cup S_{1} \cup S_{2} \cup S_{3}$, and
define
\[
A=I_0\cup I_4\cup I_{15}\cup I_{26}\cup S\cup T_{10}\cup T_{20}\cup T_{30}.
\]
Noting  that $I_0\cap S_0=\{0,k_1\}$, and 
\[
|I_\mu| = k_1 + 1,\quad 
|S_\nu| = k_1,\quad\text{and}\quad
|T_\eta| = k_1,
\]
we see that 
\[
|A| \leq 11k_1 + 2 \leq k.
\]  
We now prove that $A+A=\Z_m$.

We begin by claiming $T_a + I_b\supseteq[(a + b)k_1^2 ,  (a + b + 1)k_1^2)$. Let $n \in[(a + b)k_1^2 ,  (a + b + 1)k_1^2)$.
Then we can write $n$ as 
\[
n = (a + b) k_1^2 + ck_1 + d
\]
where $0 \leq c < k_1$ and $0 \leq d < k_1$.

If $d \geq c$, then
\[
n = ak_1^2 + c(k_1- 1) + bk_1^2 +  d - c,
\]
where $ak_1^2 +c(k_1- 1) \in T_a$, and $bk_1^2 + (d - c) \in I_b$, so $n \in T_a + I_b$. 

If $d < c$, then we must have $c \geq 1$. Hence
\[
n = ak_1^2 + (c - 1)(k_1 + 1) + bk_1^2 + k_1 + d - c + 1, 
\]
where $ak_1^2 + (c - 1)(k_1 - 1) \in T_a$, and $bk_1^2 + k_1 + d - c + 1 \in I_b$, so $n \in T_a + I_b$. 

Next we claim that  $S_a + I_b\supseteq[(a + b)k_1^2 ,  (a + b + 1)k_1^2)$. Let $n \in[(a + b)k_1^2 ,  (a + b + 1)k_1^2)$.
Then we can write $n$ as 
\[
n = (a + b) k_1^2 + ck_1 + d = ak_1^2 + ck_1 + bk_1^2 + d
\]
where  $0 \leq c < k_1$ and $0 \leq d < k_1$, such that $ak_1^2 + ck_1 \in S_a$ and $bk_1^2 + d \in I_b$, so $n \in S_a + I_b$. 

Our final claim is $T_a + S \supseteq[(a +1)k_1^2 ,  (a + 4)k_1^2)$.  Let $n \in[(a +1)k_1^2 ,  (a + 4)k_1^2)$.
Then we can write $n$ as 
\[
n = (a + b) k_1^2 + ck_1 + d, 
\]
where $1 \leq b \leq 3$, $0 \leq c < k_1$, and $0 \leq d < k_1$. 

If $c \geq d$, then
\[
n = ak_1^2 + d(k_1 + 1) + bk_1^2 + (c - d) k_1, 
\]
where $ak_1^2 + d(k_1 + 1) \in T_a$ and $bk_1^2 + (c - d)k_1 \in S_b$, so $n \in T_a + S_b$. 

If $d > c$, then
\[
n = ak_1^2 + d(k_1 + 1) + (b - 1)k_1^2 + (k_1 + c - d) k_1, 
\]
where $ak_1^2 + d(k_1 + 1) \in T_a$ and $(b - 1)k_1^2 + (k_1 + c - d)k_1 \in S_{b - 1}$, so $n \in T_a + S_{b-1}$. 



Given the properties of the modulus, we cover the following intervals by interval equivalence modulus 37$k_1^2$:  $[8k_1^2, 9k_1^2]$ , $[9k_1^2, 10k_1^2]$, and $[19k_1^2, 20k_1^2]$. 

The interval $[1, m]$ is covered as follows: 
\begin{align*}
I_0 + S &\supseteq [0, 4k_1^2),\\
I_4 + S &\supseteq [4k_1^2, 8k_1^2),\\
I_{15} + T_{30}  &\supseteq [8k_1^2, 9k_1^2),\\
I_{26} + T_{20}  &\supseteq [9k_1^2, 10k_1^2),\\
T_{10} + I_{0}  &\supseteq [10k_1^2, 11k_1^2),\\
T_{10} + S  &\supseteq [11k_1^2, 14k_1^2),\\
I_{4} + T_{10}  &\supseteq [14k_1^2, 15k_1^2),\\
I_{15} + S  &\supseteq [15k_1^2, 19k_1^2),\\
I_{26} + T_{30}  &\supseteq [19k_1^2, 20k_1^2),\\
I_{0} + T_{20}  &\supseteq [20k_1^2, 21k_1^2),\\
S + T_{20}  &\supseteq [21k_1^2, 24k_1^2),\\
I_{4} + T_{20}  &\supseteq [24k_1^2, 25k_1^2),\\
I_{15} + T_{10}  &\supseteq [25k_1^2, 26k_1^2),\\
I_{26} + S  &\supseteq [26k_1^2, 30k_1^2),\\
I_{0} + T_{30}  &\supseteq [30k_1^2, 31k_1^2),\\
S + T_{30}  &\supseteq [31k_1^2, 34k_1^2),\\
I_{4} + T_{30}  &\supseteq [34k_1^2, 35k_1^2),\\
I_{15} + T_{20}  &\supseteq [35k_1^2, 36k_1^2), \\
I_{26} + T_{10}  &\supseteq [36k_1^2, 37k_1^2).
\end{align*}

It now follows that 
\[
A + A \supseteq [0, 37k_1^2).
\]
Hence
\begin{align*}
m(2,k) &\geq 37k_1^2\\
&= 37 * \left \lfloor \frac{k - 2}{11}\right \rfloor^2\\
&> 37 \left(\frac{k - 2}{11} -1\right) ^2\\
&= \frac{37}{121}k^2-\frac{962}{121}k+\frac{6253}{121}\\
&= \frac{37}{121}k^2 + O(k)\qquad \text{ as }\quad k \to \infty. \qedhere
\end{align*}
\end{proof}
\vspace{2 mm}

\begin{theorem}
$\displaystyle m(d,4) \geq \frac{2}{243}d^4 + O(d^3)$  as $ d \to \infty$.
\end{theorem}

\begin{proof}
We now consider the following cases:

Case 1. $0 \leq x_3 < \lambda$. The following subcases need to be considered:\\
Subcase 1.a. If $2 \lambda \leq x_1 < 3 \lambda$ then $0 \leq x_0 < 2\lambda$. \\
If $0 \leq x_1 < 2 \lambda$, we have 
\[ x_0 + x_1 + x_2 + x_3 \leq 3\lambda + 2\lambda + 3\lambda + \lambda = 9\lambda \leq d, \]
which implies that $n = x_0 + x_1\alpha + x_2\beta + x_3\gamma \in dA$. \\
If $2 \lambda \leq x_1 < 3 \lambda$ and  $0 \leq x_0 < 2\lambda$, we have 
\[ x_0 + x_1 + x_2 + x_3 \leq 2\lambda + 3\lambda + 3\lambda + \lambda = 9\lambda \leq d, \]
which implies that $n = x_0 + x_1\alpha + x_2\beta + x_3\gamma \in dA$. 

Subcase 1.b. $2 \lambda \leq x_2 < 3 \lambda$, $2 \lambda \leq x_1 < 3\lambda$, $2 \lambda \leq x_0 < 3\lambda$. \\
We have
\begin{align*}
n \equiv n + m &= x_0 + x_1\alpha + x_2\beta + x_3\gamma + \lambda + \lambda \alpha + \lambda \beta + 2 \lambda \gamma\\
&=  x_0 + \lambda + (x_1 + \lambda) \alpha + (x_2 + \lambda) \beta + (x_3 + 2 \lambda) \gamma\\ 
&=  x_0 -  2\lambda + (x_1 - 2 \lambda + 1) \alpha + (x_2 + \lambda + 1) \beta + (x_3 + 2 \lambda)
 \gamma.\\ 
\end{align*}
Noting that
\[  x_0 -  2\lambda \geq 0, x_1 - 2 \lambda + 1 \geq 0, x_2 + \lambda + 1 \geq 0, \text{ and } x_3 + 2 \lambda \geq 0, \]
we see that 
\begin{align*}
x_0 -  2\lambda + x_1 - 2 \lambda + 1 + x_2 + \lambda + 1 + x_3 + 2 \lambda &= x_0  + x_1 +  x_2 + x_3 -  \lambda + 2 \\
&\leq 3 \lambda + 3 \lambda + 3 \lambda + \lambda - \lambda + 2\\
&= 9 \lambda + 2 \leq d,
\end{align*}
which implies that $n = x_0 + x_1\alpha + x_2\beta + x_3\gamma \in dA$. 

Case 2. $\lambda \leq x_3 < 2\lambda$. The following subcases need to be considered:\\
Subcase 2.a. $0 \leq x_2 < 2 \lambda.$ If $2 \lambda \leq x_1 < 3 \lambda$ then $0 \leq x_0 < 2\lambda$. \\
If $0 \leq x_1 < 2 \lambda$, we have 
\[ x_0 + x_1 + x_2 + x_3 \leq 2\lambda + 2\lambda + 2\lambda + 3\lambda = 9\lambda \leq d, \]
which implies that $n = x_0 + x_1\alpha + x_2\beta + x_3\gamma \in dA$. \\
If $2 \lambda \leq x_1 < 3 \lambda$ and  $0 \leq x_0 < 2\lambda$, we have 
\[ x_0 + x_1 + x_2 + x_3 \leq 2\lambda + 2\lambda + 3\lambda + 2 \lambda = 9\lambda \leq d, \]
which implies that $n = x_0 + x_1\alpha + x_2\beta + x_3\gamma \in dA$. 

Subcase 2.b. $\lambda \leq x_2 < 2 \lambda$, $2 \lambda \leq x_1 < 3\lambda$,  $2\lambda \leq x_0 < 3\lambda$. \\
We have
\begin{align*}
n \equiv n + m &= x_0 + x_1\alpha + x_2\beta + x_3\gamma + \lambda + \lambda \alpha + \lambda \beta + 2 \lambda \gamma\\
&=  x_0 + \lambda + (x_1 + \lambda) \alpha + (x_2 + \lambda) \beta + (x_3 + 2 \lambda) \gamma\\ 
&=  x_0 -  2\lambda + (x_1 - 2 \lambda + 1) \alpha + (x_2 + \lambda + 1) \beta + (x_3 + 2 \lambda)
 \gamma.\\ 
\end{align*}
Noting that
\[  x_0 -  2\lambda \geq 0, \  x_1 - 2 \lambda + 1 \geq 0,  \ x_2 + \lambda + 1 \geq 0, \text{ and } x_3 + 2 \lambda \geq 0, \]
we see that 
\begin{align*}
x_0 -  2\lambda + x_1 - 2 \lambda + 1 + x_2 + \lambda + 1 + x_3 + 2 \lambda &= x_0  + x_1 +  x_2 + x_3 -  \lambda + 2 \\
&\leq 3 \lambda + 3 \lambda + 2 \lambda + 2\lambda - \lambda + 2\\
&= 9 \lambda + 2 \leq d,
\end{align*}
which implies that $n = x_0 + x_1\alpha + x_2\beta + x_3\gamma \in dA$. 

Subcase 2.c. $2 \lambda \leq x_2 < 3 \lambda.$ If $ \lambda \leq x_1 < 2 \lambda$ then $0 \leq x_0 < 2\lambda$. \\
If $0 \leq x_1 < \lambda$, we have 
\[ x_0 + x_1 + x_2 + x_3 \leq 3\lambda + \lambda + 3 \lambda + 2\lambda = 9\lambda \leq d, \]
which implies that $n = x_0 + x_1\alpha + x_2\beta + x_3\gamma \in dA$. \\
If $ \lambda \leq x_1 < 2 \lambda$ and  $0 \leq x_0 < 2\lambda$, we have 
\[ x_0 + x_1 + x_2 + x_3 \leq 2\lambda + 2\lambda + 3\lambda + 2 \lambda = 9\lambda \leq d, \]
which implies that $n = x_0 + x_1\alpha + x_2\beta + x_3\gamma \in d3A$. 

Subcase 2.d. $2\lambda \leq x_2 < 3 \lambda$, $ \lambda \leq x_1 < 2\lambda$,  $2\lambda \leq x_0 < 3\lambda$. \\
We have
\begin{align*}
n \equiv n + m &= x_0 + x_1\alpha + x_2\beta + x_3\gamma + \lambda + \lambda \alpha + \lambda \beta + 2 \lambda \gamma\\
&=  x_0 + \lambda + (x_1 + \lambda) \alpha + (x_2 + \lambda) \beta + (x_3 + 2 \lambda) \gamma\\ 
&=  x_0 -  2\lambda + (x_1 + \lambda + 1) \alpha + (x_2 - 2 \lambda) \beta + (x_3 + 2 \lambda + 1)
 \gamma.\\ 
\end{align*}
Noting that
\[  x_0 -  2\lambda \geq 0,  \ x_1 + \lambda + 1 \geq 0, \  x_2 - 2 \lambda \geq 0, \text{ and } x_3 + 2 \lambda + 1 \geq 0, \]
we see that 
\begin{align*}
x_0 -  2\lambda + x_1 + \lambda + 1 + x_2  - 2 \lambda + x_3 + 2 \lambda + 1 &= x_0  + x_1 +  x_2 + x_3 -  \lambda + 2 \\
&\leq 3 \lambda + 2 \lambda + 3 \lambda + 2\lambda - \lambda + 2\\
&= 9 \lambda + 2 \leq d,
\end{align*}
which implies that $n = x_0 + x_1\alpha + x_2\beta + x_3\gamma \in dA$. 

Subcase 2.e. $2\lambda \leq x_2 < 3 \lambda$, $ 2\lambda \leq x_1 < 3\lambda$,  $0 \leq x_0 < 2\lambda$. \\
We have
\begin{align*}
n \equiv n + m &= x_0 + x_1\alpha + x_2\beta + x_3\gamma + \lambda + \lambda \alpha + \lambda \beta + 2 \lambda \gamma\\
&=  x_0 + \lambda + (x_1 + \lambda) \alpha + (x_2 + \lambda) \beta + (x_3 + 2 \lambda) \gamma\\ 
&=  x_0 +  \lambda + (x_1 - 2 \lambda) \alpha + (x_2 - 2 \lambda + 1) \beta + (x_3 + 2 \lambda + 1)\gamma.\\ 
\end{align*}
Noting that
\[  x_0 +  \lambda \geq 0,  \ x_1 - 2 \lambda \geq 0, \  x_2 - 2 \lambda \geq 0 + 1, \text{ and } x_3 + 2 \lambda + 1 \geq 0, \]
we see that 
\begin{align*}
x_0 + \lambda + x_1 - 2 \lambda + x_2  - 2 \lambda + 1 + x_3 + 2 \lambda + 1 &= x_0  + x_1 +  x_2 + x_3 -  \lambda + 2 \\
&\leq 2 \lambda + 3 \lambda + 3 \lambda + 2\lambda -  \lambda + 2\\
&= 9 \lambda + 2 \leq d,
\end{align*}
which implies that $n = x_0 + x_1\alpha + x_2\beta + x_3\gamma \in dA$. 

Subcase 2.f. $2\lambda \leq x_2 < 3 \lambda$, $ 2\lambda \leq x_1 < 3\lambda$,  $2\lambda \leq x_0 < 3\lambda$. \\
We have
\begin{align*}
n \equiv n + m &= x_0 + x_1\alpha + x_2\beta + x_3\gamma + \lambda + \lambda \alpha + \lambda \beta + 2 \lambda \gamma\\
&=  x_0 + \lambda + (x_1 + \lambda) \alpha + (x_2 + \lambda) \beta + (x_3 + 2 \lambda) \gamma\\ 
&=  x_0 -  2\lambda + (x_1 - 2 \lambda + 1) \alpha + (x_2 - 2 \lambda + 1) \beta + (x_3 + 2 \lambda + 1)\gamma.\\ 
\end{align*}
Noting that
\[  x_0 -  2\lambda \geq 0,  \ x_1 - 2 \lambda + 1 \geq 0, \  x_2 - 2 \lambda \geq 0 + 1, \text{ and } x_3 + 2 \lambda + 1 \geq 0, \]
we see that 
\begin{align*}
x_0 -  2\lambda + x_1 - 2 \lambda + 1 + x_2  - 2 \lambda + 1 + x_3 + 2 \lambda + 1 &= x_0  + x_1 +  x_2 + x_3 -  4 \lambda + 3 \\
&\leq 3 \lambda + 3 \lambda + 3 \lambda + 2\lambda - 4 \lambda + 3\\
&= 7 \lambda + 2 \leq d,
\end{align*}
which implies that $n = x_0 + x_1\alpha + x_2\beta + x_3\gamma \in dA$. 




\end{proof}
