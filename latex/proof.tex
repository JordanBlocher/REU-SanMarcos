\section{ The Proof}

Let k $\geq$ 14 be an integer. Let $k_1 = \left \lfloor \frac{k - 3}{11} \right \rfloor$. Let m = $37k_1^2$.  Define the intervals
\begin{align*}
I_0 &= [0, k_1], \\
I_4 &= [4k_1^2, 4k_1^2+k_1], \\
I_{15} &= [15k_1^2, 15k_1^2+k_1], and \\
I_{26} &= [26k_1^2, 26k_1^2+k_1] 
\end{align*}
such that I = $I_0 \cup I_4 \cup I_{15} \cup I_{26}.$

Define the sets

\begin{align*}
S_0 &= \{i(k_1 + 1) | i = 0 : k_1 - 1\},\\
S_1 &= \{k_1^2 + i(k_1 + 1) | i = 0 : k_1 - 1\},\\
S_2 &= \{2k_1^2 + i(k_1 + 1) | i = 0 : k_1 - 1\}, \text{and}\\
S_3 &= \{3k_1^2 + i(k_1 + 1) | i = 0 : k_1 - 1\}
\end{align*}
such that S = $S_0 \cup S_1 \cup S_2 \cup S_3.$

Define the sets

\begin{align*}
T_{10} &= \{k_1(10k_1 + i) | i = 0 : k_1\},\\
T_{20} &= \{k_1(20k_1 + i) | i = 0 : k_1\}, \text{and}\\
T_{30} &= \{k_1(30k_1 + i) | i = 0 : k_1\}
\end{align*}
such that T = $T_{10} \cup T_{20} \cup T_{30} .$

Let $A = I \cup S \cup T$, where $|A| = 11k_1 + 3 \leq k$. We claim that A is a 2-$basis$ for $\Z_m$.
\\ 
\begin{proof}
We begin by claiming $I_a + T_b$ $\supseteq$ $((a + b)k_1^2 ,  (a + b + 1)k_1^2]$. 

Let n $\epsilon$ $((a + b)k_1^2 ,  (a + b + 1)k_1^2]$.
Then we can write n as 
\[
n = (a + b) k_1^2 + ck_1 + d \in I_a + T_b
\]
where $0 \leq c \leq k_1$ and $0 \leq d \leq k_1$. 

Next we claim that  $I_a + S_b$ $\supseteq$ $((a + b)k_1^2 ,  (a + b + 1)k_1^2]$. 
Let n $\epsilon$ $((a + b)k_1^2 ,  (a + b + 1)k_1^2]$.
Then we can write n as 
\[
n = (a + b) k_1^2 + ck_1 + d 
\]
where  $0 \leq c < k_1$ and $0 \leq d \leq k_1$. 

case 1: d $\geq$ c
\[
n = ak_1^2 + (d - c) + bk_1^2 + c(k_1 + 1) \in I_a + S_b
\]

case 2: d $<$ c
\[
n = ak_1^2 + (k_1 + 1 + d - c) + bk_1^2 + (c - 1)(k_1 + 1) \in  I_a + S_b
\]

Our final claim is $T_a$ + S $\supseteq$ $((a +1)k_1^2 ,  (a + 4)k_1^2]$. 
Let n $\epsilon$ $((a +1)k_1^2 ,  (a + 4)k_1^2]$.
Then we can write n as 
\[
n = (a + b) k_1^2 + ck_1 + d 
\]
where $1 \leq b \leq 3$, $0 \leq c < k_1$, and $0 \leq d \leq k_1$. 

case 1: b $\geq$ c
\[
n = ak_1^2 + (b - c) + bk_1^2 + c(k_1 + 1) \in T_a + S_b
\]

case 2: b $<$ c
\[
n = ak_1^2 + (k_1 + 1 + b - c) + (b - 1)k_1^2 + c(k_1 + 1) \in  T_a + S_{b-1}
\]
Hence 
\[
n = ak_1^2 + d + bk_1^2 + ck_1 \in I_a + T_b.
\].

Note that
\begin{align*}
8 &\equiv 45 (mod\text{ }37),\\
 9 &\equiv 46 (mod\text{ }37),  \text{and} \\
19 &\equiv 56 (mod\text{ }37).
\end{align*} 
So we cover $(8k_1^2, 9k_1^2]$ by instead covering $(45k_1^2, 46k_1^2]$, and likewise for $(9k_1^2, 10k_1^2]$, and $(19k_1^2, 20k_1^2]$. 

Hence 
\begin{align*}
S + I &\supseteq [1, 8k_1^2] \cup (15k_1^2, 19k_1^2] \cup (26k_1^2, 30k_1^2],\\
I + T &\supseteq (8k_1^2, 11k_1^2] \cup (14k_1^2, 15k_1^2] \cup (19k_1^2, 21k_1^2] \cup (24k_1^2, 26k_1^2] \cup (30k_1^2, 31k_1^2] \cup (34k_1^2, 37k_1^2],\text{ and}\\
S + T &\supseteq (11k_1^2, 14k_1^2] \cup (21k_1^2, 24k_1^2] \cup (31k_1^2, 34k_1^2]. 
\end{align*}

It now follows that 
\[
A + A \supseteq [0, 37k_1^2],
\]
hence
\[
m(2,k) \geq \frac{37}{121}k^2 - O(k). 
\]


\end{proof}
