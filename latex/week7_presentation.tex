\def\sphinxdocclass{report}
\documentclass[a4paper,10pt,english]{sphinxhowto}
\usepackage{babel}
\usepackage{times}
\usepackage[Bjarne]{fncychap}
\usepackage{longtable}
\usepackage[utf8]{inputenc}
\DeclareUnicodeCharacter{00A0}{\nobreakspace}
\usepackage{sphinx}
\usepackage{dsfont}
\usepackage{braket}
\usepackage{slashed}
\usepackage{fancyvrb}
\usepackage{color}
\usepackage{float}
\usepackage{bm}
\usepackage{amsmath, amssymb, amsthm}
\usepackage{bibentry}

\let\origfigure=\figure
\renewenvironment{figure}[6]{
\origfigure[H]}
{\endlist}
\def\degrees{^\circ}
\def\d{{\rm d}}

\def\A{{\mathcal A}}
\def\h{{\mathcal h}}
\def\m{{\mathcal m}}
\def\Z{{\mathbb Z }}
\def\fslash#1{#1 \!\!\!/}
\def\F{{\bf F}}
\def\R{{\bf R}}
\def\J{{\bf J}}
\def\x{{\bf x}}
\def\y{{\bf y}}
\def\h{{\rm h}}
\def\ai{{a_{i}}}
\def\xi{{x_{i}}}
\newcommand{\bfx}{\mbox{\boldmath $x$}}
\newcommand{\bfy}{\mbox{\boldmath $y$}}
\newcommand{\bfz}{\mbox{\boldmath $z$}}
\newcommand{\bfv}{\mbox{\boldmath $v$}}
\newcommand{\bfu}{\mbox{\boldmath $u$}}
\newcommand{\bfF}{\mbox{\boldmath $F$}}
\newcommand{\bfJ}{\mbox{\boldmath $J$}}
\newcommand{\bfU}{\mbox{\boldmath $U$}}
\newcommand{\bfY}{\mbox{\boldmath $Y$}}
\newcommand{\bfR}{\mbox{\boldmath $R$}}
\newcommand{\bfg}{\mbox{\boldmath $g$}}
\newcommand{\bfc}{\mbox{\boldmath $c$}}
\newcommand{\bfxi}{\mbox{\boldmath $\xi$}}
\newcommand{\bfw}{\mbox{\boldmath $w$}}
\newcommand{\bfE}{\mbox{\boldmath $E$}}
\newcommand{\bfS}{\mbox{\boldmath $S$}}
\newcommand{\bfb}{\mbox{\boldmath $b$}}
\newcommand{\bfH}{\mbox{\boldmath $H$}}
\def\Hcurl{{\bfH({\rm curl})}}
\def\Hdiv{{\bfH({\rm div})}}

\newcommand{\dd}[2]{\frac{\partial #1}{\partial #2}}
\newcommand{\dx}{\;\mbox{d}\bfx}

\def\diag{\hbox{diag}}

\font\dsrom=dsrom10
\def\one{\hbox{\dsrom 1}}

\def\res{\mathop{\mathrm{Res}}}

\def\mathnot#1{\text{"$#1$"}}



%See Character Table for cmmib10:
%http://www.math.union.edu/~dpvc/jsmath/download/extra-fonts/cmmib10/cmmib10.html
\font\mib=cmmib10
\def\balpha{\hbox{\mib\char"0B}}
\def\bbeta{\hbox{\mib\char"0C}}
\def\bgamma{\hbox{\mib\char"0D}}
\def\bdelta{\hbox{\mib\char"0E}}
\def\bepsilon{\hbox{\mib\char"0F}}
\def\bzeta{\hbox{\mib\char"10}}
\def\boldeta{\hbox{\mib\char"11}}
\def\btheta{\hbox{\mib\char"12}}
\def\biota{\hbox{\mib\char"13}}
\def\bkappa{\hbox{\mib\char"14}}
\def\blambda{\hbox{\mib\char"15}}
\def\bmu{\hbox{\mib\char"16}}
\def\bnu{\hbox{\mib\char"17}}
\def\bxi{\hbox{\mib\char"18}}
\def\bpi{\hbox{\mib\char"19}}
\def\brho{\hbox{\mib\char"1A}}
\def\bsigma{\hbox{\mib\char"1B}}
\def\btau{\hbox{\mib\char"1C}}
\def\bupsilon{\hbox{\mib\char"1D}}
\def\bphi{\hbox{\mib\char"1E}}
\def\bchi{\hbox{\mib\char"1F}}
\def\bpsi{\hbox{\mib\char"20}}
\def\bomega{\hbox{\mib\char"21}}

\def\bvarepsilon{\hbox{\mib\char"22}}
\def\bvartheta{\hbox{\mib\char"23}}
\def\bvarpi{\hbox{\mib\char"24}}
\def\bvarrho{\hbox{\mib\char"25}}
\def\bvarphi{\hbox{\mib\char"27}}

%how to use:
%$$\alpha\balpha$$
%$$\beta\bbeta$$
%$$\gamma\bgamma$$
%$$\delta\bdelta$$
%$$\epsilon\bepsilon$$
%$$\zeta\bzeta$$
%$$\eta\boldeta$$
%$$\theta\btheta$$
%$$\iota\biota$$
%$$\kappa\bkappa$$
%$$\lambda\blambda$$
%$$\mu\bmu$$
%$$\nu\bnu$$
%$$\xi\bxi$$
%$$\pi\bpi$$
%$$\rho\brho$$
%$$\sigma\bsigma$$
%$$\tau\btau$$
%$$\upsilon\bupsilon$$
%$$\phi\bphi$$
%$$\chi\bchi$$
%$$\psi\bpsi$$
%$$\omega\bomega$$
%
%$$\varepsilon\bvarepsilon$$
%$$\vartheta\bvartheta$$
%$$\varpi\bvarpi$$
%$$\varrho\bvarrho$$
%$$\varphi\bvarphi$$

%small font
\font\mibsmall=cmmib7
\def\bsigmasmall{\hbox{\mibsmall\char"1B}}

\def\Tr{\hbox{Tr}\,}
\def\Arg{\hbox{Arg}}
\def\atan{\hbox{atan}}

\title{Extreme Orders of Additive Bases} 
\author{Jia and Friends}
\newcommand{\sphinxlogo}{}
\date{July 20, 2012}
\nobibliography*

\begin{document}

\maketitle




\section*{EXACT ORDER OF ASYMPTOTIC BASES}
\label{index:asymptotic-bases}

\textbf{Asymptotic Basis}\\
A set $\A$ such that $\ai \geq 0$ is an \emph{asymptotic basis} of order $h$ if $ \forall n (large) \in \mathbb{Z}, n = \sum_{i=1}^h$.

\textbf{\emph{Exact} Asymptotic Basis}\\
The smallest such $h$ such that the above is true.

\textbf{eg.} Let $\A = {2k + 1, k \in \mathbb{Z} }. \A $ cannot be an exact asymptotic basis of any order, as the sum of $h$ odd integers has the same parity as $h$.

\textbf{Why?}\\
Tough problem + analysis of complexity = instant gratification that is is solvable or immense sadness that it is not.

\emph{General Examples of the Results of Asymptotic Analysis}
\begin{itemize}
\item Stirling's Formula.
\item LaPlacian Methods.
\item All approximation methods ever.
\end{itemize}

\subsection*{Postage Stamp Problem}
Suppose there is only room on a letter for $n$ stamps. We have sufficient number of stamps of various denominations. What is the smallest postage that we cannot make with any n-combination of stamps?

\textbf{Analogous:} Suppose a computer cluster has enough processing power for only up to $h$ processes, each node in the cluster has processing power that can be measured as a certain number of flops. There is a queue of programs of varied complexity waiting to be run.
What is the minimum complexity that a program can take so that it is not computable on our system?
\\
We try to find the minimum set of size k such that the set of all linear combinations will not be a h-basis.
$n(h, k)$, where $ \vert \A \vert = k$, such that $\forall n \in [0, m-1], n = \sum_{i=1}^k \xi \ai$ where $\xi > 0$ and $\sum_{i=1}^k \xi \leq h$.

\newpage

\section*{EXTREMAL BASES FOR FINITE CYCLIC GROUPS}
\label{index:cyclic-groups}

\textbf{$h$-Basis}\\
$\A$ is an $h$-basis is the same is an asymptotic basis for cyclic groups.
i.e. For $a, a' \in \A, gcd(a - a') = 1$.

Similarly to the Postage Stamp Problem, we define $m(h, k)$ as our lower bound for cyclic groups.

\textbf{Why?}\\
Consider a programmable logic network (at NASA) that connects 521 memory modules to 512 processors. Note that $gcd(521, 512) = 1$.

Now consider a Cayley Graph, $\operatorname{Cay}(\mathbb{Z}_{m}, k)$ with edges from $p$ to $n$, where $k^{n} \cong p$ (mod $m$), where $k$ is a generator, $n$ is the index of $p$ relative to $k$, and $p$ is an element of $\mathbb{Z}_{m}$.

Several methods have been considered for storing arrays in a parallel memory system so that various useful partitions of an array can be fetched from the memory with a single access.

The class of \emph{p-ordered vectors} is such that adjacent elements in an unscrambled vector are p elements apart in the p-ordered vector.

Why is the logic network built by NASA so good at manipulating multi-dimensional data?

\textbf{Hint:} Each operation is equivalent to a transpose.

What are other uses for combinatorial networks?

\newpage
\section*{COMMUNICATION NETWORKS}

\textbf{Loop Network}
A network whose connectivity includes at least one Hamiltonian Cycle $\therefore$ loop networks can be described by Cayley Graphs.

\textbf{Case}
Giving finite power to each capacitor in a network, minimize the average transmission delay (minimum average distance between nodes) for a max number of nodes.

\textbf{Ad-Hoc Network}
A set of networks where all devices have equal status (on the network) and are free to link to any other device within range on the network.

\textbf{Case}
Minimize the send-recieve time of a message passed along a network with a number of recievers (nodes) and of channels (arcs) per reciever.


\newpage
\section*{SPECIAL CASES OF CAYLEY GRAPHS}

\textbf{Transposition Graphs}

$\operatorname{Cay}(\mathcal{G}, \mathcal{S})$ where $s=s^{-1}$ for any $s \in S$, i.e. $S$ is a set of transpositions. Transposition graphs $\operatorname{T}(S)$ are isomorphic if and only if their corresponding Cayley graphs are isomorphic.

If $\operatorname{T}(S)$ is edge-transitive, then $\operatorname{Cay}(\mathcal{G}, \mathcal{S})$ is arc-transitive.

\textbf{Rotational Automorphisms}

For Cayley graphs, any permutation of the generators defines a rotation scheme on any vertex.

A rotational mapping is a complete rotation, $\omega$, if and only if it is a group automorphism.

$\operatorname{Cay}(n, k)$ is a \emph{star-graph} if k and n-k are relatively prime.

$\operatorname{Cay}(\mathbb{Z}_{m}, \mathbb{Z}_{m}\setminus 0)$ has a complete rotation if and only if m is prime.

\newpage
\section*{THANKS TO:}
\label{index:thanks}

\begin{itemize}
\item \bibentry{Jia}Xingde Jia
\item \bibentry{Ray}Raymond S. Lim
\end{itemize}

\renewcommand{\indexname}{Index}
\printindex
\end{document}
