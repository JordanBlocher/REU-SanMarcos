\section{Introduction}

Let $\Gamma$ be a finite group with a subset S. The \emph{Cayley digraph}, denoted Cay($\Gamma$,A), is a digraph with vertex set $\Gamma$, such that (x,y) is a directed edge if and only if $yx^{-1}$ $\in$ A.
In this paper we will be working with $\mathbb{Z}_m$ as our vertex set, and will denote these Cayley graphs as Cay(m,A). 

 
\begin{figure}[h]
\begin{center}

\begin{tikzpicture}

\Vertex[x=170.0508492385226pt,y=257.44793707070056pt]{0};
\Vertex[x=229.43059777036808pt,y=248.64351388167475pt]{1};
\Vertex[x=256.19943380501326pt,y=195.0860839875888pt]{2};
\Vertex[x=234.95343715746685pt,y=139.06822600729095pt]{3};
\Vertex[x=176.6059109911555pt,y=124.48269654626677pt]{4};
\Vertex[x=127.63862734371301pt,y=158.71121570275264pt]{5};
\Vertex[x=124.5266485799639pt,y=218.79228696858095pt]{6};
\tikzset{EdgeStyle/.style={->}}
\Edge(1)(2)
\Edge(0)(1)
\Edge(6)(0)
\Edge(5)(6)
\Edge(4)(5)
\Edge(3)(4)
\Edge(2)(3)
\Edge(0)(2)
\Edge(2)(4)
\Edge(4)(6)
\Edge(6)(1)
\Edge(1)(3)
\Edge(3)(5)
\Edge(0)(5)
\end{tikzpicture}
\end{center}
\caption{ Cay($\mathbb{Z}_7$, \{1,2\}).}
\end{figure}

For any positive integer $d$ we define

\begin{center}
m(d,A) = max\{m | d(m,A) $\leq$ d\},
\end{center}
the largest positive integer $m$ such that the diameter, d(m,A), of the Cayley digraph Cay(m,A) is less than or equal to d. For positive integers $d$ and $k$, 
\begin{center}
m(d,k) = max\{m(d,A) | there exists a set A with |A| = k \},
\end{center}
the maximum modulus m such that there exists a generating set with cardinality equal to k and the diameter of the Cayley digraph is less than or equal to d. 

Current known bounds include
\begin{center}
m(1,k) = k+1,
\end{center}

\begin{center}
m(d,1) = d+1, and
\end{center}

\begin{center}
m(d,2) = $\lfloor$ $\frac{d(d+4)}{3}$ $\rfloor$ +1 for d $\geq$ 2. 
\end{center}

In this paper we will examine the case when k = 3. A current lower bound for this case is 
\begin{center}
m(d,3) $\geq$ $\frac{176}{2197}$$d^3$ + O($d^3$). 
\end{center}




