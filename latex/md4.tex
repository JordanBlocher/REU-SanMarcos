\section{ A Lower Bound of $m(d, 4)$.}
In 1974, Wong and Coppersmith proved that
\[m(d, k) \geq \left (\frac{d}{k} + 1\right)^k.\]
In the case when $k = 4$, this gives the lower bound to be
\[m(d, k) \geq \left (\frac{d}{4}\right )^4 + O(d^3).\]
Jia later improved this lower bound, and proved that 
\[ 
m(d, 4) \geq2.048 \left(\frac{d}{4}\right)^4  +O(d^3)\qquad\text{as}\quad d\to\infty.
\]
MENTION THE 1992 IMPROVEMENT HERE

In this paper we will show that \[m(d,4) \geq \frac{512}{243}\left(\frac d4\right)^4 + O(d^3)\approx 2.106996 \left(\frac d4\right)^4 + O(d^3).\]  This is not an improvement on the 1992 bound, but is instead given as an example of a proof of a lower bound derived from our computational algorithm.
\begin{theorem} As $d\to\infty$, 
\[
m(d,4) \geq \frac{512}{243}\left(\frac d4\right)^4 + O(d^3)\approx 2.106996 \left(\frac d4\right)^4 + O(d^3).
\]
\end{theorem}

\begin{proof}
Let $d \geq 11$ be an integer and let $\lambda = \left \lfloor \frac{d - 2}{9} \right \rfloor$. Define
\begin{align*}
\alpha &= 3 \lambda,\\
\beta &=3 \lambda \alpha,\\
\gamma &= 3 \lambda \beta,\\
m &= 2 \lambda \gamma + \lambda \beta + \lambda \alpha + \lambda.
\end{align*}
Let $A = \{1, \alpha, \beta, \gamma\}$. Then
\begin{align*}
m &= 2 \lambda d + \lambda c + \lambda b + \lambda \\
&= 54\lambda^4 + 9\lambda^3 + 3\lambda^2 + \lambda\\
&= \frac{512}{243}\left(\frac d4\right)^4 + O(d^3).
\end{align*}

Let $dA$ denote the set of all sums of at most d not necessarily distinct elements of a generating set $A$ of $\Z_m$. Then for this proof we need to show that $dA = \Z_m$ such that the Cayley digraph Cay$(m, A)$ has diameter $d(m, A) \leq d$. 

Every integer $n$ such that $0 \leq n < m$ can be expressed in the following way:
\[ n = w + x\alpha + y\beta + z\gamma\]
where 
\[ 0 \leq w \leq 3\lambda, \quad 0 \leq x \leq 3\lambda,\quad 0 \leq y \leq 3\lambda,\quad 0 \leq z \leq 2\lambda.\]
Thus we only need to show, for every $0 \leq n < m$, there exists nonnegative integers $\delta_1$, $\delta_2$, $\delta_3$, and $\delta_4$ such that we can write $n$ as
\[ n \equiv \delta_1 + \delta_2 \alpha + \delta_3 \beta + \delta_4 \gamma\]
where 
\[ \delta_1 + \delta_2 + \delta_3 + \delta_4 \leq d.\]

We now consider the following cases:

Case 1.  $0 \leq x_3 < \lambda$. The following subcases need to be considered:\\
Subcase 1.a. If $2 \lambda \leq x_1 < 3 \lambda$ then $0 \leq x_0 < 2\lambda$. \\
If $0 \leq x_1 < 2 \lambda$, we have 
\[ x_0 + x_1 + x_2 + x_3 \leq 3\lambda + 2\lambda + 3\lambda + \lambda = 9\lambda \leq d, \]
which implies that $n = x_0 + x_1\alpha + x_2\beta + x_3\gamma \in dA$. \\
If $2 \lambda \leq x_1 < 3 \lambda$ and  $0 \leq x_0 < 2\lambda$, we have 
\[ x_0 + x_1 + x_2 + x_3 \leq 2\lambda + 3\lambda + 3\lambda + \lambda = 9\lambda \leq d, \]
which implies that $n = x_0 + x_1\alpha + x_2\beta + x_3\gamma \in dA$. 

Subcase 1.b. $2 \lambda \leq x_2 < 3 \lambda$, $2 \lambda \leq x_1 < 3\lambda$, $2 \lambda \leq x_0 < 3\lambda$. \\
We have
\begin{align*}
n \equiv n + m &= x_0 + x_1\alpha + x_2\beta + x_3\gamma + \lambda + \lambda \alpha + \lambda \beta + 2 \lambda \gamma\\
&=  x_0 + \lambda + (x_1 + \lambda) \alpha + (x_2 + \lambda) \beta + (x_3 + 2 \lambda) \gamma\\ 
&=  x_0 -  2\lambda + (x_1 - 2 \lambda + 1) \alpha + (x_2 + \lambda + 1) \beta + (x_3 + 2 \lambda)
 \gamma.\\ 
\end{align*}
Noting that
\[  x_0 -  2\lambda \geq 0, x_1 - 2 \lambda + 1 \geq 0, x_2 + \lambda + 1 \geq 0, \text{ and } x_3 + 2 \lambda \geq 0, \]
we see that 
\begin{align*}
x_0 -  2\lambda + x_1 - 2 \lambda + 1 + x_2 + \lambda + 1 + x_3 + 2 \lambda &= x_0  + x_1 +  x_2 + x_3 -  \lambda + 2 \\
&\leq 3 \lambda + 3 \lambda + 3 \lambda + \lambda - \lambda + 2\\
&= 9 \lambda + 2 \leq d,
\end{align*}
which implies that $n = x_0 + x_1\alpha + x_2\beta + x_3\gamma \in dA$. 

Case 2. $\lambda \leq x_3 < 2\lambda$. The following subcases need to be considered:\\
Subcase 2.a. $0 \leq x_2 < 2 \lambda.$ If $2 \lambda \leq x_1 < 3 \lambda$ then $0 \leq x_0 < 2\lambda$. \\
If $0 \leq x_1 < 2 \lambda$, we have 
\[ x_0 + x_1 + x_2 + x_3 \leq 2\lambda + 2\lambda + 2\lambda + 3\lambda = 9\lambda \leq d, \]
which implies that $n = x_0 + x_1\alpha + x_2\beta + x_3\gamma \in dA$. \\
If $2 \lambda \leq x_1 < 3 \lambda$ and  $0 \leq x_0 < 2\lambda$, we have 
\[ x_0 + x_1 + x_2 + x_3 \leq 2\lambda + 2\lambda + 3\lambda + 2 \lambda = 9\lambda \leq d, \]
which implies that $n = x_0 + x_1\alpha + x_2\beta + x_3\gamma \in dA$. 

Subcase 2.b. $\lambda \leq x_2 < 2 \lambda$, $2 \lambda \leq x_1 < 3\lambda$,  $2\lambda \leq x_0 < 3\lambda$. \\
We have
\begin{align*}
n \equiv n + m &= x_0 + x_1\alpha + x_2\beta + x_3\gamma + \lambda + \lambda \alpha + \lambda \beta + 2 \lambda \gamma\\
&=  x_0 + \lambda + (x_1 + \lambda) \alpha + (x_2 + \lambda) \beta + (x_3 + 2 \lambda) \gamma\\ 
&=  x_0 -  2\lambda + (x_1 - 2 \lambda + 1) \alpha + (x_2 + \lambda + 1) \beta + (x_3 + 2 \lambda)
 \gamma.
\end{align*}
Noting that
\[  x_0 -  2\lambda \geq 0, \  x_1 - 2 \lambda + 1 \geq 0,  \ x_2 + \lambda + 1 \geq 0, \text{ and } x_3 + 2 \lambda \geq 0, \]
we see that 
\begin{align*}
x_0 -  2\lambda + x_1 - 2 \lambda + 1 + x_2 + \lambda + 1 + x_3 + 2 \lambda &= x_0  + x_1 +  x_2 + x_3 -  \lambda + 2 \\
&\leq 3 \lambda + 3 \lambda + 2 \lambda + 2\lambda - \lambda + 2\\
&= 9 \lambda + 2 \leq d,
\end{align*}
which implies that $n = x_0 + x_1\alpha + x_2\beta + x_3\gamma \in dA$. 

Subcase 2.c. $2 \lambda \leq x_2 < 3 \lambda.$ If $ \lambda \leq x_1 < 2 \lambda$ then $0 \leq x_0 < 2\lambda$. \\
If $0 \leq x_1 < \lambda$, we have 
\[ x_0 + x_1 + x_2 + x_3 \leq 3\lambda + \lambda + 3 \lambda + 2\lambda = 9\lambda \leq d, \]
which implies that $n = x_0 + x_1\alpha + x_2\beta + x_3\gamma \in dA$. \\
If $ \lambda \leq x_1 < 2 \lambda$ and  $0 \leq x_0 < 2\lambda$, we have 
\[ x_0 + x_1 + x_2 + x_3 \leq 2\lambda + 2\lambda + 3\lambda + 2 \lambda = 9\lambda \leq d, \]
which implies that $n = x_0 + x_1\alpha + x_2\beta + x_3\gamma \in dA$. 

Subcase 2.d. $2\lambda \leq x_2 < 3 \lambda$, $ \lambda \leq x_1 < 2\lambda$,  $2\lambda \leq x_0 < 3\lambda$. \\
We have
\begin{align*}
n \equiv n + m &= x_0 + x_1\alpha + x_2\beta + x2_3\gamma + \lambda + \lambda \alpha + \lambda \beta + 2 \lambda \gamma\\
&=  x_0 + \lambda + (x_1 + \lambda) \alpha + (x_2 + \lambda) \beta + (x_3 + 2 \lambda) \gamma\\ 
&=  x_0 -  2\lambda + (x_1 + \lambda + 1) \alpha + (x_2 - 2 \lambda) \beta + (x_3 + 2 \lambda + 1)
 \gamma.
\end{align*}
Noting that
\[  x_0 -  2\lambda \geq 0,  \ x_1 + \lambda + 1 \geq 0, \  x_2 - 2 \lambda \geq 0, \text{ and } x_3 + 2 \lambda + 1 \geq 0, \]
we see that 
\begin{align*}
x_0 -  2\lambda + x_1 + \lambda + 1 + x_2  - 2 \lambda + x_3 + 2 \lambda + 1 &= x_0  + x_1 +  x_2 + x_3 -  \lambda + 2 \\
&\leq 3 \lambda + 2 \lambda + 3 \lambda + 2\lambda - \lambda + 2\\
&= 9 \lambda + 2 \leq d,
\end{align*}
which implies that $n = x_0 + x_1\alpha + x_2\beta + x_3\gamma \in dA$. 

Subcase 2.e. $2\lambda \leq x_2 < 3 \lambda$, $ 2\lambda \leq x_1 < 3\lambda$,  $0 \leq x_0 < 2\lambda$. \\
We have
\begin{align*}
n \equiv n + m &= x_0 + x_1\alpha + x_2\beta + x_3\gamma + \lambda + \lambda \alpha + \lambda \beta + 2 \lambda \gamma\\
&=  x_0 + \lambda + (x_1 + \lambda) \alpha + (x_2 + \lambda) \beta + (x_3 + 2 \lambda) \gamma\\ 
&=  x_0 +  \lambda + (x_1 - 2 \lambda) \alpha + (x_2 - 2 \lambda + 1) \beta + (x_3 + 2 \lambda + 1)\gamma.
\end{align*}
Noting that
\[  x_0 +  \lambda \geq 0,  \ x_1 - 2 \lambda \geq 0, \  x_2 - 2 \lambda + 1 \geq 0, \text{ and } x_3 + 2 \lambda + 1 \geq 0, \]
we see that 
\begin{align*}
x_0 + \lambda + x_1 - 2 \lambda + x_2  - 2 \lambda + 1 + x_3 + 2 \lambda + 1 &= x_0  + x_1 +  x_2 + x_3 -  \lambda + 2 \\
&\leq 2 \lambda + 3 \lambda + 3 \lambda + 2\lambda -  \lambda + 2\\
&= 9 \lambda + 2 \leq d,
\end{align*}
which implies that $n = x_0 + x_1\alpha + x_2\beta + x_3\gamma \in dA$. 

Subcase 2.f. $2\lambda \leq x_2 < 3 \lambda$, $ 2\lambda \leq x_1 < 3\lambda$,  $2\lambda \leq x_0 < 3\lambda$. \\
We have
\begin{align*}
n \equiv n + m &= x_0 + x_1\alpha + x_2\beta + x_3\gamma + \lambda + \lambda \alpha + \lambda \beta + 2 \lambda \gamma\\
&=  x_0 + \lambda + (x_1 + \lambda) \alpha + (x_2 + \lambda) \beta + (x_3 + 2 \lambda) \gamma\\ 
&=  x_0 -  2\lambda + (x_1 - 2 \lambda + 1) \alpha + (x_2 - 2 \lambda + 1) \beta + (x_3 + 2 \lambda + 1)\gamma.
\end{align*}
Noting that
\[  x_0 -  2\lambda \geq 0,  \ x_1 - 2 \lambda + 1 \geq 0, \  x_2 - 2 \lambda + 1 \geq 0, \text{ and } x_3 + 2 \lambda + 1 \geq 0, \]
we see that 
\begin{align*}
x_0 -  2\lambda + x_1 - 2 \lambda + 1 + x_2  - 2 \lambda + 1 + x_3 + 2 \lambda + 1 &= x_0  + x_1 +  x_2 + x_3 -  4 \lambda + 3 \\
&\leq 3 \lambda + 3 \lambda + 3 \lambda + 2\lambda - 4 \lambda + 3\\
&= 7 \lambda + 2 \leq d,
\end{align*}
which implies that $n = x_0 + x_1\alpha + x_2\beta + x_3\gamma \in dA$. 

Case 3. $2 \lambda \leq x_3 < 3 \lambda$. The following subcases need to be considered:\\
Subcase 3.a. $0 \leq x_2 <  \lambda.$ If $2 \lambda \leq x_1 < 3 \lambda$ then $0 \leq x_0 < 2\lambda$. \\
If $0 \leq x_1 < 2\lambda$, we have 
\[ x_0 + x_1 + x_2 + x_3 \leq 3 \lambda + 2 \lambda + \lambda + 3\lambda = 9\lambda \leq d, \]
which implies that $n = x_0 + x_1\alpha + x_2\beta + x_3\gamma \in dA$. \\
If $2 \lambda \leq x_1 < 3 \lambda$ and  $0 \leq x_0 < 2\lambda$, we have 
\[ x_0 + x_1 + x_2 + x_3 \leq 2\lambda + 3\lambda + \lambda + 3 \lambda = 9\lambda \leq d, \]
which implies that $n = x_0 + x_1\alpha + x_2\beta + x_3\gamma \in dA$. 

Subcase 3.b. $0 \leq x_2 <  \lambda$, $ 2\lambda \leq x_1 < 3\lambda$,  $2 \lambda \leq x_0 < 3\lambda$. \\
We have
\begin{align*}
n \equiv n + m &= x_0 + x_1\alpha + x_2\beta + x_3\gamma + \lambda + \lambda \alpha + \lambda \beta + 2 \lambda \gamma\\
&=  x_0 + \lambda + (x_1 + \lambda) \alpha + (x_2 + \lambda) \beta + (x_3 + 2 \lambda) \gamma\\ 
&=  x_0 -  2\lambda + (x_1 - 2 \lambda + 1) \alpha + (x_2 + \lambda + 1) \beta + (x_3 + 2 \lambda )\gamma.
\end{align*}
Noting that
\[  x_0 -  2\lambda \geq 0,  \ x_1 - 2 \lambda + 1 \geq 0, \  x_2 + \lambda + 1 \geq 0, \text{ and } x_3 + 2 \lambda \geq 0, \]
we see that 
\begin{align*}
x_0 -  2\lambda + x_1 - 2 \lambda + 1 + x_2  + \lambda + 1 + x_3 + 2 \lambda &= x_0  + x_1 +  x_2 + x_3 -  \lambda + 2 \\
&\leq 3 \lambda + 3 \lambda + \lambda + 3\lambda - \lambda + 2\\
&= 9 \lambda + 2 \leq d,
\end{align*}
which implies that $n = x_0 + x_1\alpha + x_2\beta + x_3\gamma \in dA$.

Subcase 3.c. $\lambda \leq x_2 < 2 \lambda$ and $0 \leq x_1 < 2\lambda$.  If $\lambda \leq x_1 < 2 \lambda$ then $0 \leq x_0 < \lambda$. \\
If $0 \leq x_1 < \lambda$, we have 
\[ x_0 + x_1 + x_2 + x_3 \leq 3\lambda + \lambda + 2\lambda + 3\lambda = 9\lambda \leq d, \]
which implies that $n = x_0 + x_1\alpha + x_2\beta + x_3\gamma \in dA$. \\
If $ \lambda \leq x_1 < 2 \lambda$ and  $0 \leq x_0 < \lambda$, we have 
\[ x_0 + x_1 + x_2 + x_3 \leq \lambda + 2\lambda + 2 \lambda + 3 \lambda = 8\lambda \leq d, \]
which implies that $n = x_0 + x_1\alpha + x_2\beta + x_3\gamma \in dA$.

Thus we have shown that every element $0 \leq n < m$ is contained in $dA$, which implies the diameter of the Cayley digraph Cay$(\Z_m, A)$ is less than or equal to $d$. 
Hence,
\[
m(d, 4) \geq2.048 \left(\frac{d}{4}\right)^4  +O(d^3)\qquad\text{as}\quad d\to\infty. \qedhere
\]
\end{proof}
